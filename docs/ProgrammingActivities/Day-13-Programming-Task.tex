\documentclass[]{tufte-handout}

% ams
\usepackage{amssymb,amsmath}

\usepackage{ifxetex,ifluatex}
\usepackage{fixltx2e} % provides \textsubscript
\ifnum 0\ifxetex 1\fi\ifluatex 1\fi=0 % if pdftex
  \usepackage[T1]{fontenc}
  \usepackage[utf8]{inputenc}
\else % if luatex or xelatex
  \makeatletter
  \@ifpackageloaded{fontspec}{}{\usepackage{fontspec}}
  \makeatother
  \defaultfontfeatures{Ligatures=TeX,Scale=MatchLowercase}
  \makeatletter
  \@ifpackageloaded{soul}{
     \renewcommand\allcapsspacing[1]{{\addfontfeature{LetterSpace=15}#1}}
     \renewcommand\smallcapsspacing[1]{{\addfontfeature{LetterSpace=10}#1}}
   }{}
  \makeatother
\fi

% graphix
\usepackage{graphicx}
\setkeys{Gin}{width=\linewidth,totalheight=\textheight,keepaspectratio}

% booktabs
\usepackage{booktabs}

% url
\usepackage{url}

% hyperref
\usepackage{hyperref}

% units.
\usepackage{units}


\setcounter{secnumdepth}{-1}

% citations

% pandoc syntax highlighting
\usepackage{color}
\usepackage{fancyvrb}
\newcommand{\VerbBar}{|}
\newcommand{\VERB}{\Verb[commandchars=\\\{\}]}
\DefineVerbatimEnvironment{Highlighting}{Verbatim}{commandchars=\\\{\}}
% Add ',fontsize=\small' for more characters per line
\newenvironment{Shaded}{}{}
\newcommand{\KeywordTok}[1]{\textcolor[rgb]{0.00,0.44,0.13}{\textbf{{#1}}}}
\newcommand{\DataTypeTok}[1]{\textcolor[rgb]{0.56,0.13,0.00}{{#1}}}
\newcommand{\DecValTok}[1]{\textcolor[rgb]{0.25,0.63,0.44}{{#1}}}
\newcommand{\BaseNTok}[1]{\textcolor[rgb]{0.25,0.63,0.44}{{#1}}}
\newcommand{\FloatTok}[1]{\textcolor[rgb]{0.25,0.63,0.44}{{#1}}}
\newcommand{\ConstantTok}[1]{\textcolor[rgb]{0.53,0.00,0.00}{{#1}}}
\newcommand{\CharTok}[1]{\textcolor[rgb]{0.25,0.44,0.63}{{#1}}}
\newcommand{\SpecialCharTok}[1]{\textcolor[rgb]{0.25,0.44,0.63}{{#1}}}
\newcommand{\StringTok}[1]{\textcolor[rgb]{0.25,0.44,0.63}{{#1}}}
\newcommand{\VerbatimStringTok}[1]{\textcolor[rgb]{0.25,0.44,0.63}{{#1}}}
\newcommand{\SpecialStringTok}[1]{\textcolor[rgb]{0.73,0.40,0.53}{{#1}}}
\newcommand{\ImportTok}[1]{{#1}}
\newcommand{\CommentTok}[1]{\textcolor[rgb]{0.38,0.63,0.69}{\textit{{#1}}}}
\newcommand{\DocumentationTok}[1]{\textcolor[rgb]{0.73,0.13,0.13}{\textit{{#1}}}}
\newcommand{\AnnotationTok}[1]{\textcolor[rgb]{0.38,0.63,0.69}{\textbf{\textit{{#1}}}}}
\newcommand{\CommentVarTok}[1]{\textcolor[rgb]{0.38,0.63,0.69}{\textbf{\textit{{#1}}}}}
\newcommand{\OtherTok}[1]{\textcolor[rgb]{0.00,0.44,0.13}{{#1}}}
\newcommand{\FunctionTok}[1]{\textcolor[rgb]{0.02,0.16,0.49}{{#1}}}
\newcommand{\VariableTok}[1]{\textcolor[rgb]{0.10,0.09,0.49}{{#1}}}
\newcommand{\ControlFlowTok}[1]{\textcolor[rgb]{0.00,0.44,0.13}{\textbf{{#1}}}}
\newcommand{\OperatorTok}[1]{\textcolor[rgb]{0.40,0.40,0.40}{{#1}}}
\newcommand{\BuiltInTok}[1]{{#1}}
\newcommand{\ExtensionTok}[1]{{#1}}
\newcommand{\PreprocessorTok}[1]{\textcolor[rgb]{0.74,0.48,0.00}{{#1}}}
\newcommand{\AttributeTok}[1]{\textcolor[rgb]{0.49,0.56,0.16}{{#1}}}
\newcommand{\RegionMarkerTok}[1]{{#1}}
\newcommand{\InformationTok}[1]{\textcolor[rgb]{0.38,0.63,0.69}{\textbf{\textit{{#1}}}}}
\newcommand{\WarningTok}[1]{\textcolor[rgb]{0.38,0.63,0.69}{\textbf{\textit{{#1}}}}}
\newcommand{\AlertTok}[1]{\textcolor[rgb]{1.00,0.00,0.00}{\textbf{{#1}}}}
\newcommand{\ErrorTok}[1]{\textcolor[rgb]{1.00,0.00,0.00}{\textbf{{#1}}}}
\newcommand{\NormalTok}[1]{{#1}}

% longtable

% multiplecol
\usepackage{multicol}

% strikeout
\usepackage[normalem]{ulem}

% morefloats
\usepackage{morefloats}


% tightlist macro required by pandoc >= 1.14
\providecommand{\tightlist}{%
  \setlength{\itemsep}{0pt}\setlength{\parskip}{0pt}}

% title / author / date
\title{In-Class Programming Activity 13}
\author{Math 253: Statistical Computing \& Machine Learning}
\date{k-1 cross-validation}


\begin{document}

\maketitle




Today you are going to write a function to carry out k-fold cross
validation. Recall that this involves dividing the available data set
into k equal-sized, independent sets. For each of the k sets, reserve
that set as testing data, fit your model to the remaining data, and
evaluate the fitted model on the reserved set. In doing this, you will
accumulate k different estimates of the model error. Average those to
produce an overall estimate of model error.

\section{A function framework}\label{a-function-framework}

Call your function \texttt{k\_fold1()}. It should have this interface:

\begin{Shaded}
\begin{Highlighting}[]
\NormalTok{k_fold1 <-}\StringTok{ }\NormalTok{function(formula, }\DataTypeTok{method =} \NormalTok{lm, }\DataTypeTok{data =} \NormalTok{mtcars, }\DataTypeTok{predfun =} \NormalTok{predict, }\DataTypeTok{k =} \DecValTok{10}\NormalTok{) \{}
  \CommentTok{# do the calculations, }
  \CommentTok{# producing an estimate of error}
  \KeywordTok{return}\NormalTok{(error_estimate)}
\NormalTok{\}}
\end{Highlighting}
\end{Shaded}

As written above, the function does nothing. You're going to fix that.
But use exactly the interface (function name, argument names and order,
default values) given above.

\section{The k sets}\label{the-k-sets}

In k-fold cross validation, you'll divide the cases in your data into k
sets. To help do that, you're going to create a vector named
\texttt{sets} that has one element for each case in the data. The value
of that element will be 1 if the corresponding case is to be in set 1 of
the k sets, 2 if the case is to be in set 2, and so on. It doesn't
matter how you assign cases to sets, so long as there are k sets of
roughly equal size.

There are many ways to construct the vector \texttt{sets}.
\marginnote{\texttt{k <- 10}\\\noindent\texttt{sets <- (1:51 \%\% k) + 1}\\\noindent\texttt{\# or, alternatively, ...}\\\noindent\texttt{sets <- rep(1:k, each = 51/k,}\\\noindent\texttt{                        length.out = 51)}}
Here are two operations, \texttt{\%\%} and \texttt{rep(,\ ,\ each=)}
that might be the basis for two different ways of creating the vector
Here are two operations that may give you a hint. The example assumes
that there are 51 rows in the \texttt{data}.

\section{The loop}\label{the-loop}

Your function will perform the same operation k times. Doing this will
produce k numbers, the mean square prediction error for each of the k
test data sets. In writing a loop, you \ldots{}

\marginnote{Here's a code fragment that handles (1), (2), and (4).

\noindent -----------
}

\begin{enumerate}
\def\labelenumi{\arabic{enumi}.}
\tightlist
\item
  Set up the
  state\marginnote{\texttt{mspe <- numeric(k)}\\\noindent\texttt{for (i in 1:k) \{     }\\\noindent\texttt{\# Your statements go here}\\\noindent\texttt{\}}\\\noindent\texttt{mean(mspe)}}
  or ``accumulator'' that will be updated as you go through the loop
\item
  Construct the outline of the loop
\item
  Fill in the operations to be carried out inside the loop.
  \marginnote[0cm]{\_\_\_\_\_\_\_\_\_\_\_\_\_\_
  \newline \noindent Make sure that you understand what \texttt{numeric(k)} is doing.}
\item
  Tidy up the accumulator to produce the final result.
\end{enumerate}

\section{Inside the loop}\label{inside-the-loop}

As you can see, the statements inside the loop will be evaluated k
times. The object \texttt{i} can be used inside the loop to indicate
which of these k passes through the loop is currently being performed.

Here's what to do inside the loop:

\begin{enumerate}
\def\labelenumi{\arabic{enumi}.}
\tightlist
\item
  Compare \texttt{i} to \texttt{sets} to create a logical (that is,
  \texttt{TRUE}/\texttt{FALSE} vector) of the rows to be used for the
  \textbf{test} set. Create a new data frame, \texttt{For\_Testing} that
  has just these rows.
\item
  Using logical negation (that is, \texttt{!}) to create another data
  frame, fill a data frame \texttt{For\_Training} with the remaining
  rows.
\item
  Fit your model using \texttt{For\_Training}. For simplicity in
  developing your function, you can use a particular model appropriate
  for the \texttt{mtcars} data set:
  \texttt{mod\ \textless{}-\ lm(mpg\ \textasciitilde{}\ hp\ +\ wt\ +\ am)}.
  Note that \texttt{mpg} is the response variable. Later on, you'll
  replace this with a more general statement.
\item
  Evaluate the model on the \texttt{For\_Testing} data. You can do this
  with
  \texttt{pred\_vals\ \textless{}-\ predict(mod,\ newdata\ =\ For\_Testing)}
\item
  Calculate the mean square prediction error (MSPE). This will be
  \texttt{mean((For\_Testing{[}{[}"mpg"{]}{]}\ -\ pred\_vals)\^{}2\ )}
\item
  Save the MSPE in the appropriate slot of \texttt{mspe}.
\end{enumerate}

\section{Trying out your function}\label{trying-out-your-function}

At this point, you should have a function \texttt{k\_fold1()} that, when
evaluated with the default settings for the arguments, returns a number.
That number should be very roughly similar in size to this one, which is
called the ``in-sample'' error.

\begin{verbatim}
## [1] 5.634096
\end{verbatim}

In-sample prediction errors are biased to be lower than cross-validated
prediction errors.

\section{Generalizing the function}\label{generalizing-the-function}

When you are satisfied that your \texttt{k\_fold1()} function is
working, \emph{copy} the code to create another function which we'll
call \texttt{k\_fold()}. This is going to be the generalization of the
function to other data sets and other model formulas.

Modify \texttt{k\_fold()} in the following ways:

\begin{itemize}
\tightlist
\item
  Replace \texttt{mpg\ \textasciitilde{}\ hp\ +\ wt\ +\ am} with
  \texttt{formula}. This will allow you to specify the formula on as the
  first argument.
\item
  Replace \texttt{lm()} with \texttt{method()}. This will allow your
  function to use modeling types other than \texttt{lm()}
\item
  Replace \texttt{predict()} with \texttt{predfun()}
\item
  Replace \texttt{{[}{[}"mpg"{]}{]}} with
  \texttt{{[}{[}as.character(formula{[}{[}2{]}{]}){]}{]}}.\marginnote[0cm]{Indexing formula objects enables extraction of parts of the formula.  For those with an interest in computer languages, a formula can be used as a parse tree, with the operator (e.g. \texttt{\~} or \texttt{+}, etc.) at the top and the arguments underneath.}
\end{itemize}

Test out your function using the same data and formula as before, that
is:

\begin{Shaded}
\begin{Highlighting}[]
\KeywordTok{k_fold}\NormalTok{(}\DataTypeTok{formula =} \NormalTok{mpg ~}\StringTok{ }\NormalTok{hp +}\StringTok{ }\NormalTok{wt +}\StringTok{ }\NormalTok{am, }\DataTypeTok{data=}\NormalTok{mtcars)}
\end{Highlighting}
\end{Shaded}

You should get the same answer as with \texttt{k\_fold1()}.



\end{document}
